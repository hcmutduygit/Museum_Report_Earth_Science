\chapter{Mineral Resources (First Floor)}

\section{Concepts of Mineral Resources}

\subsection{Energy minerals}
Energy minerals include fossil fuels that power modern civilization. Crude oil (petroleum) forms from organic-rich marine sediments buried and heated over millions of years, migrating into reservoir rocks and trapped by impermeable cap rocks. The museum displays oil shale, tar sands, and petroleum reservoir core samples. Coal represents terrestrial plant matter compressed and metamorphosed through peat, lignite, bituminous, and anthracite grades as temperature and pressure increase with burial depth. Exhibits show coal rank progression, vitrinite reflectance, and associated mining methods. Vietnam's coal resources concentrated in the Quang Ninh basin produce anthracite for domestic and export markets. Natural gas deposits often accompany petroleum in sedimentary basins. Energy mineral exhibits emphasize exploration techniques (seismic surveys, well logging), extraction technology, refining processes, and the transition toward renewable energy sources to address climate change concerns.
\begin{figure}[H]
        \centering
        \includegraphics[width=0.65\textwidth]{pictures/chapter2/c2_p07_ks_nhienlieu1.png}
        \caption{Energy mineral samples displayed in the museum.}
    \end{figure}
    \begin{figure}[H]
        \centering
        \includegraphics[width=0.65\textwidth]{pictures/chapter2/c2_p08_ks_nhienlieu2.png}
        \caption{Energy mineral samples displayed in the museum.}
    \end{figure}
\subsection{Metallic minerals}
Metallic mineral exhibits showcase ore deposits that supply industrial metals essential for technology and infrastructure. Iron ores (hematite, magnetite, goethite) dominate global metal production for steel manufacturing, with Vietnam's titaniferous iron sands mined from coastal placers. Titanium minerals (ilmenite, rutile) serve aerospace and pigment industries. Aluminum derives from bauxite laterites formed by tropical weathering of aluminum-rich rocks; Vietnam's Central Highlands host significant bauxite reserves. Copper sulfide ores (chalcopyrite, bornite) provide electrical wiring and electronics materials. Lead-zinc deposits often co-occur in volcanogenic massive sulfides and carbonate-hosted replacement bodies. Tin (cassiterite) concentrates in greisen zones and alluvial placers associated with granite intrusions. Tungsten (wolframite, scheelite) and molybdenum (molybdenite) form in high-temperature hydrothermal systems. Gold occurs in quartz veins, epithermal deposits, and placer accumulations. Rare earth elements and specialty metals support advanced technologies. The displays explain hydrothermal ore-forming processes, magmatic differentiation, and weathering concentration mechanisms that create economic deposits.
    \begin{figure}[H]
        \centering
        \includegraphics[width=0.65\textwidth]{pictures/chapter2/c2_p09_ks_kimloai1.png}
        \caption{Metallic mineral samples displayed in the museum.}
    \end{figure}
        \begin{figure}[H]
        \centering
        \includegraphics[width=0.65\textwidth]{pictures/chapter2/c2_p10_ks_kimloai2.png}
        \caption{Metallic mineral samples displayed in the museum.}
    \end{figure}
        \begin{figure}[H]
        \centering
        \includegraphics[width=0.65\textwidth]{pictures/chapter2/c2_p11_ks_kimloai3.png}
        \caption{Metallic mineral samples displayed in the museum.}
    \end{figure}
\subsection{Non-metallic minerals}
Non-metallic industrial minerals support diverse applications in construction, manufacturing, and agriculture. Kaolin (kaolinite clay) forms from feldspar weathering and serves ceramic, paper coating, and refractory industries. Glass sand requires high-purity quartz with minimal iron content for optical and container glass production. Silica sand feeds silicon metal and semiconductor manufacturing. Asbestos fibers (chrysotile, amphiboles) historically provided heat-resistant materials but now face restrictions due to health hazards. Apatite and phosphorite supply phosphorus for fertilizers, originating from marine sedimentary enrichment or igneous carbonatite complexes. Pyrite (iron sulfide) produces sulfuric acid for chemical processes. Phosphorite deposits in Vietnam occur in marine Cambrian formations. Construction materials include limestone for cement and aggregate, gypsum for wallboard, clay for bricks and tiles, and dimension stone (granite, marble) for building facades. The exhibits connect mineral properties (hardness, chemical stability, thermal tolerance) to industrial specifications and emphasize sustainable mining practices, environmental remediation, and recycling initiatives.
    \begin{figure}[H]
        \centering
        \includegraphics[width=0.65\textwidth]{pictures/chapter2/c2_p12_ks_kokimloai1.png}
        \caption{Non-metallic mineral samples displayed in the museum.}
    \end{figure}
        \begin{figure}[H]
        \centering
        \includegraphics[width=0.65\textwidth]{pictures/chapter2/c2_p13_ks_kokimloai2.png}
        \caption{Non-metallic mineral samples displayed in the museum.}
    \end{figure}
        \begin{figure}[H]
        \centering
        \includegraphics[width=0.65\textwidth]{pictures/chapter2/c2_p14_ks_kokimloai3.png}
        \caption{Non-metallic mineral samples displayed in the museum.}
    \end{figure}
\subsection{Precious and semi-precious stones}
Gemstone exhibits feature crystalline minerals valued for beauty, rarity, and durability. Ruby (red corundum) and sapphire (blue and other colored corundum) achieve hardness 9 on the Mohs scale, second only to diamond, and form in metamorphic marbles and basaltic magmas. Vietnam's Luc Yen district in Yen Bai Province produces rubies and sapphires from marble-hosted deposits. Topaz crystals occur in pegmatites and greisen alteration zones associated with granitic intrusions. Quartz varieties include amethyst (purple), citrine (yellow-orange), rose quartz (pink), and smoky quartz (brown-gray), forming in hydrothermal veins and geodes. Other gemstones displayed may include beryl varieties (emerald, aquamarine), tourmaline, garnet, spinel, jade (jadeite, nephrite), and zircon. The exhibits explain optical properties (refraction, dispersion, pleochroism), crystal habits, inclusion characteristics used for identification and origin determination, and value factors (color saturation, clarity, cut quality, carat weight). Gemstone formation connects to plate tectonics, with metamorphic, magmatic, and hydrothermal environments producing distinct assemblages. Ethical sourcing and gemstone treatment disclosure are also addressed.
    \begin{figure}[H]
        \centering
        \includegraphics[width=0.65\textwidth]{pictures/chapter2/c2_p15_ks_daquy1.png}
        \caption{Precious and semi-precious stone samples displayed in the museum.}
    \end{figure}
        \begin{figure}[H]
        \centering
        \includegraphics[width=0.65\textwidth]{pictures/chapter2/c2_p16_ks_daquy2.png}
        \caption{Precious and semi-precious stone samples displayed in the museum.}
    \end{figure}
\subsection{Mineral water}
Mineral water resources represent groundwater naturally enriched with dissolved minerals from subsurface rock interactions. The exhibits describe hydrogeological systems where precipitation infiltrates permeable strata, circulates through aquifers, and dissolves soluble minerals (calcium, magnesium, sodium, potassium, bicarbonate, sulfate, chloride, silica). Thermal springs occur where geothermal gradients or volcanic heat sources warm circulating groundwater, bringing deep-seated minerals to the surface. Vietnam's mineral springs include those in mountainous regions where fracture systems facilitate deep circulation. Chemical analyses displayed show total dissolved solids (TDS), ion concentrations, pH, and trace elements that determine water classification (carbonate, sulfate, chloride types). Beneficial health properties attributed to specific mineral compositions support spa and bottled water industries. The displays also address groundwater recharge zones, aquifer vulnerability to contamination, sustainable extraction rates, and regulatory standards for potable and therapeutic mineral water. Connections to geological structure, lithology, and hydrologic cycling emphasize the importance of protecting groundwater resources from pollution and overexploitation.
    \begin{figure}[H]
        \centering
        \includegraphics[width=0.65\textwidth]{pictures/chapter2/c2_p17_nckhoang.png}
        \caption{Mineral water samples displayed in the museum.}
    \end{figure}
\section{Concepts Related to Geological Resources}

Geological resources encompass naturally occurring materials in Earth's crust that possess economic value and support human societies. These resources form through geological processes operating over millions of years and concentrate in specific geological environments controlled by tectonics, magmatism, sedimentation, and weathering. Resource classification distinguishes reserves (economically extractable with current technology) from resources (total in-ground quantity regardless of economic viability). Exploration methods integrate geological mapping, geophysics (seismic, gravity, magnetic, electrical surveys), geochemistry (stream sediment sampling, soil analysis), and remote sensing to identify prospective areas. Drilling programs and sampling confirm deposit characteristics before mining feasibility studies assess grade, tonnage, metallurgy, infrastructure requirements, environmental impacts, and market conditions.

Sustainable resource management balances extraction rates with conservation, environmental protection, and social responsibility. Environmental considerations include mine waste disposal, acid mine drainage prevention, habitat restoration, and greenhouse gas emissions reduction. Recycling and circular economy principles extend resource lifespans by recovering metals and materials from end-of-life products. Resource security concerns involve supply chain vulnerabilities, geopolitical factors, and strategic stockpiling of critical materials. The exhibits emphasize Vietnam's geological heritage and responsible development of mineral wealth to support economic growth while safeguarding natural ecosystems and community well-being for future generations.
