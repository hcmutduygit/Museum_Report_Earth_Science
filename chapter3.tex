\chapter{Minerals (Second Floor)}

\section{Concepts of Minerals}

Minerals are naturally occurring, inorganic crystalline solids with definite chemical compositions and ordered atomic structures. Each mineral species exhibits characteristic physical properties arising from its crystal lattice arrangement and chemical bonding. The museum's mineral collection illustrates fundamental concepts essential for identification and classification.

Crystal structure defines the three-dimensional arrangement of atoms in repeating patterns, classified into seven crystal systems (cubic, tetragonal, orthorhombic, monoclinic, triclinic, hexagonal, trigonal) based on symmetry elements and unit cell parameters. Crystal habits describe external forms: prismatic, tabular, acicular (needle-like), bladed, equant, or massive. Cleavage represents preferential breakage planes along weak atomic bonds, characterized by quality (perfect, good, poor) and number of directions. Fracture describes breakage patterns without cleavage: conchoidal (smooth curved surfaces), uneven, splintery, or hackly.

Hardness measures resistance to scratching, quantified by Mohs scale (1=talc through 10=diamond). Streak color appears when a mineral is powdered by rubbing on unglazed porcelain. Luster describes light reflection quality: metallic (opaque, mirror-like), submetallic, vitreous (glassy), resinous, pearly, silky, greasy, or dull. Specific gravity compares density to water, reflecting atomic mass and packing efficiency. Color varies due to chemical composition (idiochromatic minerals) or trace impurities and defects (allochromatic minerals). Transparency ranges from transparent through translucent to opaque.

Chemical composition expressed by molecular formulas indicates constituent elements and their ratios. Isomorphism allows atomic substitution within a crystal structure (e.g., iron-magnesium replacement in olivine solid solutions). Polymorphism describes minerals with identical chemistry but different structures (e.g., graphite and diamond both pure carbon). Mineral genesis relates formation to geological environments: igneous crystallization from cooling magma, sedimentary precipitation from aqueous solutions, metamorphic recrystallization under elevated temperature and pressure, and hydrothermal deposition from hot fluids.

\section{Classification of Minerals}

Minerals are systematically classified based on chemical composition and crystal structure, primarily using the Dana and Strunz classification systems. The chemical classification groups minerals by anionic components, reflecting bonding characteristics and formation processes.

\textbf{Native Elements:} Pure elements uncombined with other substances, including metals (gold, silver, copper, platinum), semi-metals (arsenic, antimony, bismuth), and non-metals (sulfur, graphite, diamond). These form under reducing conditions or as weathering products.

\textbf{Sulfides and Sulfosalts:} Compounds of metals with sulfur (and related elements selenium, tellurium, arsenic, antimony). Major ore minerals include pyrite (FeS$_2$), galena (PbS), sphalerite (ZnS), chalcopyrite (CuFeS$_2$), molybdenite (MoS$_2$), and arsenopyrite (FeAsS). These dominate hydrothermal vein deposits and volcanogenic massive sulfides.

\textbf{Oxides and Hydroxides:} Metal-oxygen compounds including simple oxides (hematite Fe$_2$O$_3$, magnetite Fe$_3$O$_4$, rutile TiO$_2$, corundum Al$_2$O$_3$), multiple oxides (spinel MgAl$_2$O$_4$, ilmenite FeTiO$_3$), and hydroxides (goethite FeO(OH), gibbsite Al(OH)$_3$, brucite Mg(OH)$_2$). These form through igneous crystallization, metamorphism, and supergene weathering processes.

\textbf{Halides:} Halogen-bearing minerals including chlorides (halite NaCl, sylvite KCl), fluorides (fluorite CaF$_2$), and bromides/iodides. These typically precipitate from evaporating brines in arid environments or hydrothermal solutions.

\textbf{Carbonates:} Metal-carbonate compounds with (CO$_3$)$^{2-}$ anionic groups. Common species include calcite (CaCO$_3$), dolomite (CaMg(CO$_3$)$_2$), magnesite (MgCO$_3$), siderite (FeCO$_3$), rhodochrosite (MnCO$_3$), smithsonite (ZnCO$_3$), and malachite (Cu$_2$CO$_3$(OH)$_2$). Carbonates dominate marine sedimentary environments and oxidized ore zones.

\textbf{Sulfates:} Metal-sulfate minerals with (SO$_4$)$^{2-}$ groups including gypsum (CaSO$_4 \cdot$2H$_2$O), anhydrite (CaSO$_4$), barite (BaSO$_4$), celestine (SrSO$_4$), and anglesite (PbSO$_4$). These form through evaporation or oxidation of sulfide ores.

\textbf{Phosphates, Arsenates, and Vanadates:} Compounds with (PO$_4$)$^{3-}$, (AsO$_4$)$^{3-}$, or (VO$_4$)$^{3-}$ groups. Important phosphates include apatite (Ca$_5$(PO$_4$)$_3$(F,Cl,OH)), essential for fertilizers and biological systems. Arsenates and vanadates occur in oxidized ore zones.

\textbf{Silicates:} The largest mineral class, comprising approximately 90\% of Earth's crust by volume. Silicates contain silicon-oxygen tetrahedra (SiO$_4$)$^{4-}$ linked in various configurations. Subclasses include:
\begin{itemize}
\item \textbf{Nesosilicates (isolated tetrahedra):} olivine ((Mg,Fe)$_2$SiO$_4$), garnet group, zircon (ZrSiO$_4$)
\item \textbf{Sorosilicates (double tetrahedra):} epidote group, hemimorphite
\item \textbf{Cyclosilicates (ring structures):} beryl (Be$_3$Al$_2$Si$_6$O$_{18}$), tourmaline group
\item \textbf{Inosilicates (chains):} pyroxene group (single chains), amphibole group (double chains)
\item \textbf{Phyllosilicates (sheets):} mica group (muscovite, biotite), clay minerals (kaolinite, montmorillonite), serpentine, talc, chlorite
\item \textbf{Tectosilicates (three-dimensional frameworks):} quartz (SiO$_2$), feldspar group (orthoclase, plagioclase), feldspathoid group, zeolite group
\end{itemize}

This classification reflects increasing polymerization of silicate tetrahedra and correlates with igneous crystallization sequences, metamorphic grade, and weathering stability.

\section{Mineral Display Cases}
\subsection{Minerals containing metallic elements}
This section showcases ore minerals that supply industrial metals. Displays are organized by metal type and include specimens illustrating primary ores, associated gangue minerals, and typical paragenetic assemblages.

\textbf{Iron Ore Minerals:} Hematite specimens show characteristic steel-gray to red-brown coloration with red streak and metallic to earthy luster. Magnetite exhibits strong magnetic properties and black color. Goethite and limonite represent hydrous iron oxides formed through weathering. Siderite (iron carbonate) appears in sedimentary iron formations. Pyrite (FeS$_2$), while not an ore, demonstrates brassy yellow metallic luster and cubic crystal forms.
     \begin{figure}[H]
          \centering
          \includegraphics[width=0.85\textwidth]{pictures/chapter3/c3_p01_quangsat.png}
          \caption{Iron ore mineral samples displayed in the museum.}
     \end{figure}
\textbf{Copper Minerals:} Native copper displays ductile metallic characteristics. Chalcopyrite (CuFeS$_2$) shows brassy yellow color and is the primary copper ore. Bornite (Cu$_5$FeS$_4$) exhibits iridescent tarnish (peacock ore). Chalcocite (Cu$_2$S) forms in supergene enrichment zones. Oxidized secondary minerals include malachite (green, banded), azurite (deep blue), chrysocolla (blue-green), and cuprite (red).
     \begin{figure}[H]
          \centering
          \includegraphics[width=0.85\textwidth]{pictures/chapter3/c3_p02_quangdong.png}
          \caption{Copper ore mineral samples displayed in the museum.}
     \end{figure}
\textbf{Lead-Zinc Minerals:} Galena (PbS) demonstrates perfect cubic cleavage, high density, and metallic gray luster. Sphalerite (ZnS) varies from yellow-brown to black with resinous luster. Cerussite (PbCO$_3$) and anglesite (PbSO$_4$) form as secondary lead minerals. Smithsonite (ZnCO$_3$) and hemimorphite (Zn$_4$Si$_2$O$_7$(OH)$_2 \cdot$H$_2$O) represent zinc carbonates and silicates.
     \begin{figure}[H]
          \centering
          \includegraphics[width=0.85\textwidth]{pictures/chapter3/c3_p03_quangkem.png}
          \caption{Lead-zinc ore mineral samples displayed in the museum.}
     \end{figure}
\textbf{Tin and Tungsten Minerals:} Cassiterite (SnO$_2$) exhibits adamantine luster and high specific gravity. Wolframite ((Fe,Mn)WO$_4$) and scheelite (CaWO$_4$) serve as tungsten ores. These typically occur in pegmatites and greisen alteration zones associated with granitic intrusions.
     \begin{figure}[H]
          \centering
          \includegraphics[width=0.85\textwidth]{pictures/chapter3/c3_p04_quangthiec.png}
          \caption{Tin and tungsten ore mineral samples displayed in the museum.}
     \end{figure}
\textbf{Aluminum Minerals:} Bauxite ore consists of gibbsite (Al(OH)$_3$), boehmite (AlO(OH)), and diaspore, formed through tropical weathering. Corundum (Al$_2$O$_3$) appears as gemstone varieties (ruby, sapphire) and industrial abrasive.
     \begin{figure}[H]
          \centering
          \includegraphics[width=0.85\textwidth]{pictures/chapter3/c3_p05_quangnhom.png}
          \caption{Aluminum ore mineral samples displayed in the museum.}
     \end{figure}
\textbf{Titanium Minerals:} Ilmenite (FeTiO$_3$) and rutile (TiO$_2$) concentrate in heavy mineral placers and mafic igneous rocks. Titanite (sphene, CaTiSiO$_5$) occurs in intermediate igneous and metamorphic rocks.
     \begin{figure}[H]
          \centering
          \includegraphics[width=0.85\textwidth]{pictures/chapter3/c3_p06_quangtitan.png}
          \caption{Titanium ore mineral samples displayed in the museum.}
     \end{figure}
\textbf{Gold and Silver Minerals:} Native gold specimens demonstrate malleability and characteristic yellow color. Electrum represents natural gold-silver alloys. Silver sulfides include argentite (Ag$_2$S) and proustite (Ag$_3$AsS$_3$, ruby silver).
     \begin{figure}[H]
          \centering
          \includegraphics[width=0.85\textwidth]{pictures/chapter3/c3_p07_quangvang.png}
          \caption{Gold ore mineral samples displayed in the museum.}
     \end{figure}
Each display case includes information on chemical composition, crystal system, physical properties, geological occurrence, economic significance, and processing methods.

\subsection{Minerals containing silica}
Silicate minerals dominate Earth's crust and exhibit remarkable chemical and structural diversity. The displays are organized by silicate structure type, progressing from isolated tetrahedra to three-dimensional frameworks.

\textbf{Nesosilicates:} Olivine series specimens show gradation from forsterite (Mg$_2$SiO$_4$, pale green) to fayalite (Fe$_2$SiO$_4$, dark olive). Garnet group representatives include pyrope (Mg$_3$Al$_2$Si$_3$O$_{12}$, red), almandine (Fe$_3$Al$_2$Si$_3$O$_{12}$, deep red), grossular (Ca$_3$Al$_2$Si$_3$O$_{12}$, various colors), and andradite (Ca$_3$Fe$_2$Si$_3$O$_{12}$, green-black). Zircon (ZrSiO$_4$) displays adamantine luster and serves for radiometric dating. Topaz (Al$_2$SiO$_4$(F,OH)$_2$) exhibits perfect basal cleavage.

\textbf{Inosilicates:} Pyroxene group specimens include augite ((Ca,Na)(Mg,Fe,Al,Ti)(Si,Al)$_2$O$_6$), diopside (CaMgSi$_2$O$_6$), enstatite (MgSiO$_3$), and jadeite (NaAlSi$_2$O$_6$). These show two cleavage directions intersecting at approximately 90 degrees. Amphibole group minerals include hornblende ((Ca,Na)$_{2-3}$(Mg,Fe,Al)$_5$(Al,Si)$_8$O$_{22}$(OH)$_2$), tremolite (Ca$_2$Mg$_5$Si$_8$O$_{22}$(OH)$_2$), and actinolite (Ca$_2$(Mg,Fe)$_5$Si$_8$O$_{22}$(OH)$_2$), characterized by cleavages intersecting at approximately 120 degrees.

\textbf{Phyllosilicates:} Mica specimens demonstrate perfect basal cleavage producing flexible sheets. Muscovite (KAl$_2$(AlSi$_3$O$_{10}$)(OH)$_2$) appears colorless to pale, while biotite (K(Mg,Fe)$_3$(AlSi$_3$O$_{10}$)(OH)$_2$) exhibits dark brown-black coloration. Clay minerals including kaolinite (Al$_2$Si$_2$O$_5$(OH)$_4$) display earthy textures. Chlorite group minerals show green coloration. Talc (Mg$_3$Si$_4$O$_{10}$(OH)$_2$) represents the softest mineral (hardness 1). Serpentine minerals include chrysotile (asbestos variety) and antigorite.

\textbf{Tectosilicates:} Quartz (SiO$_2$) displays extensive variety: rock crystal (clear), amethyst (purple), citrine (yellow), rose quartz (pink), smoky quartz (brown), and milky quartz (white). Chalcedony microcrystalline varieties include agate (banded), jasper (opaque, various colors), carnelian (red-orange), and onyx (black and white banded). Feldspar group minerals include orthoclase (KAlSi$_3$O$_8$) with characteristic pink color and plagioclase series from albite (NaAlSi$_3$O$_8$) to anorthite (CaAl$_2$Si$_2$O$_8$) showing progressive compositional variation. Feldspathoid minerals like nepheline ((Na,K)AlSiO$_4$) occur in silica-undersaturated igneous rocks. Zeolite group minerals including natrolite, stilbite, and heulandite demonstrate microporous structures useful for ion exchange and molecular sieving applications.

Each silicate display includes crystal models showing tetrahedral linkages, thin-section photomicrographs revealing optical properties, and explanations of how structure controls physical characteristics and geological stability.
     \begin{figure}[H]
          \centering
          \includegraphics[width=0.65\textwidth]{pictures/chapter3/c3_p08_coich.png}
          \caption{Another mineral samples displayed in the museum.}
     \end{figure}