\usepackage[english]{babel} % English language
\usepackage[utf8]{inputenc} % UTF-8 encoding
\usepackage{graphicx} % Chèn hình ảnh
\usepackage{fancyhdr} % Gói hỗ trợ tạo header và footer fancy
\usepackage{changepage} % Thay đổi lề

% Chèn code
\usepackage{listings} % Thêm gói listings để chèn code
\usepackage{xcolor} % Màu cho code
\lstset{
    language=Python,
    basicstyle=\small\ttfamily,
    numbers=left,
    numberstyle=\tiny\color{gray},
    stepnumber=1,
    numbersep=8pt,
    tabsize=4,
    breaklines=true,
    breakatwhitespace=false,
    xleftmargin=1.5em,
    framexleftmargin=1.5em,
    keywordstyle=\color{blue}\bfseries,
    commentstyle=\color{green!60!black}\itshape,
    stringstyle=\color{orange!80!red},
    identifierstyle=\color{black},
    emphstyle=\color{red},
    frame=single,
    rulecolor=\color{black!30},
    backgroundcolor=\color{gray!5},
    showstringspaces=false,
    captionpos=b,
    morekeywords={True, False, None, self, cv2, np, plt},
    emph={range, len, print, open, str, int, float, list, dict, enumerate, zip},
    emphstyle=\color{purple!80!blue},
}

% Footnote and References
\usepackage[style=numeric,backend=biber]{biblatex} % Sử dụng gói biblatex
\usepackage{capt-of} %  Footnote trong caption
\usepackage[perpage]{footmisc} % Đánh số lại chú thích mỗi trang

% Thiết lập bảng
\usepackage{array} % Gói hỗ trợ các bảng phức tạp
\usepackage{tabularx}
\usepackage{longtable} % Tạo bảng qua nhiều trang
\usepackage{cellspace}
\usepackage{diagbox} % Gói hỗ trợ tạo các ô chéo trong bảng
\usepackage{multirow}
\usepackage{makecell}
\usepackage{xcolor}
\usepackage{colortbl}
\usepackage[export]{adjustbox}

% Thiết lập công thức toán học
\usepackage{amsmath} % Gói hỗ trợ các công thức toán học
\usepackage{amsfonts} % Gói hỗ trợ các ký hiệu toán học
\usepackage{amssymb} % Gói hỗ trợ các ký hiệu toán học
\usepackage{graphicx} % Gói hỗ trợ chèn hình ảnh
\usepackage{bm} % Chữ in đậm trong công thức toán 

% Đồ thị 
\usepackage{pgfplots}
\pgfplotsset{compat=1.18}

% Thiết lập thêm tùy chọn cho dấu gạch đầu dòng
\usepackage{enumitem}

% Thiết lập khác
\usepackage{tikz}
\usepackage{color}
\usepackage{subcaption}
\usepackage{framed}
\usepackage{float} % Để chèn hình ảnh vào đúng vị trí
\usepackage{fancyvrb} % Đưa dữ liệu dạng nguyên thủy vào

% Thiết lập kích thước
\usepackage{geometry}
\geometry{
    left=3cm,
    right=2cm,
    top=2.5cm,
    bottom=2.5cm,
}
\usepackage{hyperref} %Chèn link
\hypersetup{urlcolor=black,linkcolor=black,citecolor=black,colorlinks=true} % Màu cho các đường nét
\everymath{\color{black}}
\pagestyle{fancy}

% Đánh số theo section
\numberwithin{equation}{section}  % Công thức: (1.1), (1.2), (2.1)...
\numberwithin{figure}{section}    % Hình: Figure 1.1, Figure 1.2...
\numberwithin{table}{section}     % Bảng: Table 1.1, Table 1.2...
\AtBeginDocument{%
    \counterwithin{lstlisting}{section}
}