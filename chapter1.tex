\chapter{General Geology (Ground Floor)}

\section{Geological Evolution According to the Old Geochronological Scale}

\subsection{Precambrian Eon: from 4 billion years ago to 570 million years ago}
\begin{itemize}
    \item \textbf{Archean Eon}: spanning from about 4.0 to 2.5 billion years ago and lasting roughly 2.6 billion years, was the period during which the early Earth cooled and developed a stable crust and oceans. It was also the time when the first simple life forms, mainly bacteria, appeared and began influencing the planet’s early environment.  
    \item \textbf{Proterozoic Eon}: \\[0.3cm]
    Following the Archean Eon, the Proterozoic began about 2.6 billion years ago and lasted for roughly 2 billion years (from 2.6 to 0.6 billion years ago).

    In Vietnam, Proterozoic formations are commonly found in several uplift zones, including the Song Hong, Phan Si Pan, Song Chay, Song Ma, Phu Hoat, and Kon Tum regions. They include:
    \begin{itemize}
        \item Paleoproterozoic crystalline rocks from the Song Hong series (Northeast Vietnam), the Xuan Dai series (Northwest Vietnam), and the Ngoc Linh series (Kon Tum zone).
        \item Mesoproterozoic metamorphic rocks classified within the Muong Xen series (Northern Vietnam).
        \item Neoproterozoic metamorphic rocks, often present in units overlying the Lower Cambrian and containing talc and microphyton-bearing quartzite. These belong to the Song Chay (Northern Vietnam), Sa Pa, Nam Co (Northwest Vietnam), Bu Khang (North-Central Vietnam), and Yen Se (Kon Tum zone) formations.
    \end{itemize}
    \begin{figure}[H]
        \centering
        \includegraphics[width=0.75\textwidth]{pictures/p1_proterozoic.png}
        \caption{Proterozoic rock formations in Vietnam.}
    \end{figure}
\end{itemize}
\subsection{Phanerozoic Eon: from 570 million years ago to present}
\begin{itemize}
    \item \textbf{Paleozoic Era} \\[0.3cm]
    The Paleozoic Era began about 600 million years before present, lasting approximately 375 million years (600--225 Ma). It is divided into six periods: Cambrian, lasting about 100 million years (600--500 Ma); Ordovician, 60 million years (500--440 Ma); Silurian, 40 million years (440--400 Ma); Devonian, 60 million years (400--350 Ma); Carboniferous, 80 million years (350--270 Ma); and Permian, 45 million years (270--225 Ma).

    Lower Cambrian formations usually show close relations with Neoproterozoic formations. Middle Cambrian--Lower Ordovician sediments are widely distributed and vary in composition depending on their geographic areas. In northern North Bac Bo, West Bac Bo, and North Trung Bo, they are mainly composed of carbonates interbedded with terrigenous sediments containing fossils of Trilobites and articulate Brachiopods. In North and East Bac Bo, they form monotypic volcanic–terrigenous sequences bearing fossils of compositions completely different from those in the regions mentioned above.

    Ordovician and Silurian formations are exposed in many areas of North and East Bac Bo. They exhibit close relationships within continuous stratigraphic sequences that include Cambrian and Lower Devonian sedimentary layers. In North Trung Bo, these formations consist of terrigenous sediments containing deep-dwelling Graptolites together with Corals, Crinoids, and Brachiopods. In East and North Bac Bo, as well as in North Trung Bo, they include volcano-terrigenous sediments bearing deep-dwelling Graptolites along with Corals, Crinoids, and Brachiopods.
    \begin{figure}[H]
        \centering
        \includegraphics[width=0.75\textwidth]{pictures/p2_paleozoic.png}
        \caption{Paleozoic rock formations in Vietnam.}
    \end{figure}
    \item \textbf{Mesozoic Era}: \\[0.3cm]
    The Mesozoic Era began 225 million years before present days with a duration of 155 million years (225--70 Ma). It is composed of 3 periods: Triassic having a duration of 45 million years (225--180 Ma), Jurassic -- 45 million years (180--135 Ma), and Cretaceous -- 65 million years (135--70 Ma). During Mesozoic through two great stages: Triassic before Norian and Norian--Jurassic--Cretaceous.

    Before Norian the Triassic is characterized mainly by marine sediments with the intercalation of volcanogenic formations. At the end of the stage the marine regime in some areas was replaced by the continental regime. In the An Chau and Song Da basins there are all the formations from Lower Triassic to Upper Triassic. At the same time in the Song Hinh areas and West Nam Bo there are only Lower and Middle Triassic formations. Triassic sediments are rich in fossils, such as Bivalves, Ammonoids, Gastropods, Brachiopods, Hexacorals and Plants.

    Norian--Jurassic--Cretaceous stage is represented mainly by coal-bearing and red continental formations (in the North), or continental terrigenous intercalated with Jurassic marine formations (in the South). In the South, the fossils of this stage in long and complete sequences are largely distributed; in some places forming great massifs of volcanics (in Tam Lang and Tu Le). From Jurassic the sedimentary differentiation is not big. In all the Viet Nam territory red continental formations are largely developed which contain in abundance Plants (Norian--Rhaetian), Ammonites and Bivalves (Jurassic and Cretaceous).
    \begin{figure}[H]
        \centering
        \includegraphics[width=0.75\textwidth]{pictures/p3_mesozoic.png}
        \caption{Mesozoic rock formations in Vietnam.}
    \end{figure}
    \item \textbf{Cenozoic Era}: \\[0.3cm]
    The Cenozoic Era began about 70 million years before present and continues to the present day. It is composed of three periods: Paleogene lasting 45 million years (70--25 Ma), Neogene -- 23.2 million years (25--1.8 Ma), and Quaternary lasting from 1.8 million years ago to the present.

    In Viet Nam, Cenozoic formations are diverse. Paleogene is extensively developed; the upper Paleogene is widely distributed in Northwest Viet Nam (in the Pu Tra region) and Northeast Viet Nam (in the Na Vien area). Neogene is the main oil–gas bearing unit of the entire region, with its sedimentary basins found in both the North and South. Neogene sediments are composed mainly of sand, kaolin, bentonite, red clay, and conglomerates in the North, while in Central Viet Nam they form thick and weakly consolidated beds associated with volcanic activity. Neogene is very rich in fossils exhibiting a wide range of facies, including plant remains, molluscs, and planktonic organisms such as Ostracoda, Foraminifera, and vertebrates.
    \begin{figure}[H]
        \centering
        \includegraphics[width=0.75\textwidth]{pictures/p4_cenozoic.png}
        \caption{Cenozoic rock formations in Vietnam.}
    \end{figure}
\end{itemize}

\section{Geological Processes: Introduction to 6 Processes}

\subsection{Cosmic Processes}
Displaying meteorites and explaining their origin, composition, and significance in understanding the early formation of the Solar System and the Earth's primordial materials.
\begin{figure}[H]
        \centering
        \includegraphics[width=0.75\textwidth]{pictures/chapter2/c2_p01_vutru.png}
        \caption{Cosmic rock samples displayed in the museum.}
    \end{figure}
\subsection{Tectonic Processes}
Showing tectonic maps, folded and faulted rock samples, and illustrations of deformation and contact metamorphism. This process highlights how plate movements shape continents, create mountains, and influence seismic and volcanic activities.
\begin{figure}[H]
        \centering
        \includegraphics[width=0.75\textwidth]{pictures/chapter2/c2_p02_gneis_biotit.png}
        \caption{Tectonic rock samples displayed in the museum.}
    \end{figure}
\subsection{Magmatic Processes}
Displaying intrusive and extrusive igneous rocks, volcanic rocks, and intrusive magma bodies. These exhibits demonstrate how magma forms, rises, cools, and crystallizes, providing insight into volcanic eruptions and the formation of the Earth's crust.
\begin{figure}[H]
        \centering
        \includegraphics[width=0.75\textwidth]{pictures/chapter2/c2_p03_granit_biotit.png}
        \caption{Granite biotite rock sample with biotite minerals displayed in the museum.}
    \end{figure}
    \begin{figure}[H]
        \centering
        \includegraphics[width=0.75\textwidth]{pictures/chapter2/c2_p04_granodiorit.png}
        \caption{Granodiorite rock samples displayed in the museum.}
    \end{figure}
\subsection{Metamorphic Processes}
Displaying metamorphic rocks and diagrams explaining contact metamorphism and hydrothermal processes. This section emphasizes how heat, pressure, and chemically active fluids transform existing rocks into new types with different textures and mineral compositions.
    \begin{figure}[H]
        \centering
        \includegraphics[width=0.75\textwidth]{pictures/chapter2/c2_p05_dahoa_bienchat.png}
        \caption{Metamorphic rock samples displayed in the museum.}
    \end{figure}
\subsection{Sedimentary Processes}
Displaying sedimentary rocks, clastic rocks, and chemical and biochemical sedimentary rocks. The process illustrates how weathering, erosion, transportation, deposition, and lithification create layered rocks that preserve valuable information about past environments.
    \begin{figure}[H]
        \centering
        \includegraphics[width=0.65\textwidth]{pictures/chapter2/c2_p06_tramtich.png}
        \caption{Sedimentary rock samples displayed in the museum.}
    \end{figure}
\subsection{Weathering Processes}
Displaying samples of chemical and physical weathering. This process shows how rocks break down under atmospheric conditions, influencing soil formation, landscape evolution, and material cycling on Earth's surface.
